\documentclass[a4paper]{article}
%review


%#########################################################################
\usepackage[utf8]{inputenc}
\usepackage[T1]{fontenc}
%\usepackage[francais]{babel}
%\usepackage{lmodern}
\usepackage[hmargin=4cm, vmargin=2cm, includeheadfoot]{geometry}
\usepackage{alltt}
\usepackage{multicol}
\usepackage{amsmath,amssymb}
\usepackage{color}
\usepackage{graphicx}
\usepackage[francais,bloc,completemulti]{automultiplechoice}     % Mandatory for conversion
\usepackage{mhchem} % needed for chemical equations
% needed by amc to be commented for amc2moodle usage (fp is not yet supported)
%\usepackage{fp} 
% exemple de commande utilisateur
\providecommand{\abs}[1]{\lvert#1\rvert}
%#########################################################################
% Entête
%#########################################################################



\title{Conversion test for AMC to moodle}
\author{amc2moodle}



%#########################################################################
% Document
%#########################################################################
\begin{document}

%
% B A R E M E
% e=incohérence; b=bonne; m=mauvaise; p planché (on ne descent pas en dessous)
\baremeDefautS{e=-0.5,b=1,m=-0.5}% never put b<1,
\baremeDefautM{e=-0.5,b=1,m=-0.25,p=-0.5}% never put b<1, with amc2moodle m correspond to the grade if all the wrong answers are ticked, b correspond to the grade if all the good answers are ticked


\element{dyna}{
% Test for long equation that may breakdown in LaTeXML due to '%\n' insertion
    \begin{question}{long_eq}
        \`A partir des résultats précédents, donner l'expression dans la base $(\vec{x}_1,\vec{y}_1,\vec{z}_0)$ de l'accélération du point $G_3$ lié à $S_3$ dans son mouvement par rapport à $R_0$:
        \[\overrightarrow{\Gamma_{({G_3},S_3/R_0)}}\]
        \begin{reponses}
            \mauvaise{$(\ddot{\lambda}-\lambda\dot{\theta}_{1}^{2})\vec{x}_{1}+(\lambda\ddot{\theta}_{1}+2\dot{\lambda}\dot{\theta}_{1})\vec{y}_{1}$}
            \mauvaise{$\ddot{\lambda}\vec{x}_{1}+\lambda\ddot{\theta}_{1}\vec{y}_{1}$}
            \bonne{$\ddot{\lambda}\vec{x}_{1}$}
            \mauvaise{$\ddot{\lambda}\vec{x}_{1}+(\lambda\ddot{\theta}_{1}+\dot{\lambda}\dot{\theta}_{1})\vec{y}_{1}$}
        \end{reponses}
    \end{question}
}






% #################################################################
% C R E A T I O N  D E S  C O P I E S
% #################################################################
\exemplaire{1}{        % nombre de sujet différent

  %debut de l'en-tête des copies :
  \vspace*{.5cm}
  \begin{minipage}{.4\linewidth}
    \centering\large\texttt{amc2moodle}
  \end{minipage}
  \champnom{\fbox{
      \begin{minipage}{.5\linewidth}
        Nom et prénom :

        \vspace*{.5cm}\dotfill
        \vspace*{1mm}
      \end{minipage}
    }}

  \begin{flushleft}
    Test for long equation.
    \begin{center}
      \Large{\textsc{QCM using AMC Latex Format}}\\
      \normalsize
    \end{center}
  \end{flushleft}

  \cleargroup{BigGroupe}
  \copygroup{dyna}{BigGroupe}
  \restituegroupe{BigGroupe}
}




\end{document}
