\documentclass[a4paper]{article}
% -------------------------::== package ==::---------------------------
\usepackage[utf8]{inputenc}
\usepackage[T1]{fontenc}
\usepackage{alltt}
\usepackage{multicol}
\usepackage{amsmath,amssymb}
\usepackage{color}
\usepackage{graphicx}
% Mandatory for conversion
\usepackage[francais,bloc,completemulti]{automultiplechoice}
\usepackage{tikz}
\usepackage{hyperref}
\usepackage{ulem} % strike text
% fp is needed by AMC for numerical question with float. Need to be commented for amc2moodle usage (fp is not yet supported)
\usepackage{fp} 

% -----------------------::== newcommand ==::--------------------------
\newcommand{\feedback}[1]{}
\begin{document}

% -----------------------------------------------------------------------------
\element{Défaut pour BN1-moodle2amc}{
  \begin{question}{Essai}\label{q:Essai}   
    Explain in few words the aim of this course.
% It is possible to add more granularity with partially correct answer using \wrongchoice[P]{p}\scoring{0.5}
\AMCOpen{lines=3}{    \correctchoice[OK]{OK}    \wrongchoice[F]{F}}
  \end{question}
}

% -----------------------------------------------------------------------------
\element{Défaut pour BN1-moodle2amc}{
  \begin{question}{html layout}\label{q:html layout}   
    a link \href{https://github.com/nennigb/amc2moodle}{here } and an image \includegraphics[]{./Figures/4.png} and an equation \( \int_{2\pi} x^2 \mathrm{d} x \)
\begin{center}
    centered text
\end{center}
flush left text
\begin{flushright}
    flush right text
\end{flushright}
In moodle editor, there is also \textsubscript{exponent} and \textsuperscript{indice} and \sout{that}and svg file \includegraphics[width=100px]{./Figures/dessin.png} 
  \begin{choices}
	    \correctchoice{This is the good \underline{underlined} answer.}    \wrongchoice{This is one \textit{italic} wrong answer.}    \wrongchoice{This a wrong \textbf{bold} answer.}    \wrongchoice{This a wrong \textbf{strong} answer.}    \wrongchoice{This a wrong \emph{emphasis} answer.}
  \end{choices}

  \end{question}
}

% -----------------------------------------------------------------------------
\element{Défaut pour BN1-moodle2amc}{
  \begin{question}{table}\label{q:table}   
    Test html table conversion to tex        
\begin{center}
	
  \begin{tabular}{cccc}
  \hline
   & weight & width & length\\   
  \hline
sys1 & 1 kg & 0.35 m & 1 m\\   
sys2 & 2 kg & - & 1.5 m\\   
  \hline
	
  \end{tabular}\\
 table legend
\end{center}
 
  \begin{choices}
	    \wrongchoice{wrong answer}    \correctchoice{the good answer is obviously a weird table        
\begin{center}
	
  \begin{tabular}{cc}
  \hline
  stuff1 & stuff2\\   
  \hline
\(x^2\) & bold \textbf{text}\\   
  \hline
	
  \end{tabular}\\
 
\end{center}
 }    \wrongchoice{an other table, more simple         
\begin{center}
	
  \begin{tabular}{ccc}
  \hline
    Firstname & Lastname & Age\\   
  Jill & Smith & 50\\   
  Eve & Jackson & 94\\   
  \hline
	
  \end{tabular}\\
 
\end{center}
 }
  \end{choices}

  \end{question}
}

% -----------------------------------------------------------------------------
\element{Défaut pour BN1-moodle2amc}{
  \begin{question}{ProblemDescription}\label{q:ProblemDescription}   
    \QuestionIndicative
Provide a \textbf{description} of a problem that can be common to several questions. It is useful to define notation, pictures, equations \(\int_0^1 x \mathrm{d} x = 0\)...
  \end{question}
}

% -----------------------------------------------------------------------------
\element{Défaut pour BN1-moodle2amc-num}{
  \begin{questionmultx}{num:int}\label{q:num:int}   
    Find \(x\) such \(2x-300=0\) ? Here \(x\) is an integer, test for \textbf{exact} match, only.
% need \usepackage{fp} for floatting point computation 
\AMCnumericChoices{150}{sign = false, exact = 0, digits = 3, decimals = 0, scoreexact = 1}   

  \end{questionmultx}
}

% -----------------------------------------------------------------------------
\element{Défaut pour BN1-moodle2amc-num}{
  \begin{questionmultx}{num:float}\label{q:num:float}   
    Give an approximated value for \(\pi\) up to 3 digits?
% need \usepackage{fp} for floatting point computation 
\AMCnumericChoices{3.141592653589793}{sign = false, exact = 1, digits = 4, decimals = 3, scoreexact = 1}   

  \end{questionmultx}
}

% -----------------------------------------------------------------------------
\element{Défaut pour BN1-moodle2amc-num}{
  \begin{questionmultx}{num:2ans}\label{q:num:2ans}   
    Let \(z=3+2\mathrm{i}\). What is the imaginary part ?
% need \usepackage{fp} for floatting point computation 
\AMCnumericChoices{2}{sign = false, exact = 0, digits = 1, decimals = 0, scoreexact = 1}   

  \end{questionmultx}
}

% -----------------------------------------------------------------------------
\element{Défaut pour BN1-moodle2amc-num}{
  \begin{questionmultx}{num:2rounding}\label{q:num:2rounding}   
    Give an approximated value for \(\sqrt{2}\) up to 3 digits ?
% need \usepackage{fp} for floatting point computation 
\AMCnumericChoices{1.4142135623730951}{scoreapprox = 0.5, digits = 4, sign = false, decimals = 3, approx = 100, exact = 1, scoreexact = 1}   

  \end{questionmultx}
}

% ============================================================================
\exemplaire{1}{    	% nombre de sujet différent

% Replace with your Header
\vspace*{.5cm}
\begin{minipage}{.4\linewidth}
    \centering\large\bf Test
\end{minipage}
\champnom{\fbox{
    \begin{minipage}{.5\linewidth}
Nom et prénom :

\vspace*{.5cm}\dotfill
\vspace*{1mm}
    \end{minipage}
}}

\begin{center}
  \Large{\textsc{An AMC quiz generated from moodle XML questions export}}\\
  \normalsize
\end{center}

% mélange et catégorie (groupe dans ACM)
\cleargroup{allquestions}
\copygroup{Défaut pour BN1-moodle2amc}{allquestions}
\copygroup{Défaut pour BN1-moodle2amc-num}{allquestions}
% Shuffling is commented for testing
%\melangegroupe{allquestions}
\restituegroupe{allquestions}
}
\end{document}